\section{Limitations of \textit{in vitro}-based investigations 

Many of the experiments conducted in order to complete this study were performed using \textit{in vitro} systems. These systems were designed to interrogate force activity by holding the force threshold constant (e.g. micropillars, DNA membrane tension probes}, while observing the cell's activity in time. It would be interesting to investigate dynamics of synaptic force exertion in a single cell

While these arrangements are useful for ease of investigation, holding constant a variable as broad as the force threshold creates major scientific blind spots. This is true particularly in studies of the effector cell, in which the arrangement ignores half of the cellular conversation: the target cell. It would fascto dynamically resist possibility that  

Of course, many of these experiments were also performed using the OT-I transgene system. 

- Relevance of our in vitro system? Stiffnesses? The limitations of nterrogating a moleucle's biomechanology independently of its biochemistry?