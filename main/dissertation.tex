\documentclass[phd,tocprelim]{cornell}
%
% tocprelim option must be included to put the roman numeral pages in the
% table of contents
%
% The cornellheadings option will make headings completely consistent with
% guidelines.
%
% This template was originally provided by Blake Jacquot, and
% fixed up by Andrew Myers.
%
%Some possible packages to include
\usepackage[utf8]{inputenc}
\usepackage{graphicx,pstricks}
\usepackage{microtype}
\usepackage{graphics}
\usepackage{moreverb}
\usepackage{subfigure}
\usepackage{epsfig}
\usepackage{subfigure}
\usepackage{hangcaption}
\usepackage{txfonts}
\usepackage{palatino}
% Packages added by Xihe:
\usepackage{subfiles}
\usepackage{bm}
\usepackage[font=small, labelfont=bf]{caption}
\usepackage{array} % for flexible tables
\usepackage{hyperref} % for links
\usepackage{amsbsy} % for bold
\usepackage{caption}
\usepackage{indentfirst} % indent first paragraph after \section tags
\usepackage{booktabs} % for tables

%if you're having problems with overfull boxes, you may need to increase
%the tolerance to 9999
\tolerance=9999

\bibliographystyle{unsrt}
%\bibliographystyle{IEEEbib}

\renewcommand{\caption}[1]{\singlespacing\hangcaption{#1}\normalspacing}
\renewcommand{\topfraction}{0.85}
\renewcommand{\textfraction}{0.1}
\renewcommand{\floatpagefraction}{0.75}

\title{MECHANOTRANSDUCTION OF THE CYTOTOXIC T LYMPHOCYTE  EFFECTOR RESPONSE}
\author {Mitchell S. Wang}
\conferraldate {May}{2022}
\degreefield {Ph.D.}
\copyrightholder{Mitchell S. Wang}
\copyrightyear{2022}

\begin{document}

\maketitle
\makecopyright

\begin{abstract}
\subfile{../abstract/abstract}
\end{abstract}

\begin{biosketch}
Mitchell was born in Palo Alto, California in 1993, and grew up in the suburbs of Philadelphia, Pennsylvania. After high school, he attended New York University, where he studied for his Bachelor of Arts degree as a double major in both biology and economics. During his time in college, he studied Alzheimer’s disease in the laboratory of Dr. Yueming Li at Memorial Sloan Kettering Cancer Center. Upon graduating in 2015 and with an interest in the biomedical sciences, he entered the Pharmacology program at Weill Cornell Graduate School of Medical Sciences and joined the laboratory of Dr. Morgan Huse at Memorial Sloan Kettering Cancer Center for his thesis work, where he has since studied the principles of mechanical force exertion in T cells and their contribution to cellular cytotoxicity.
\end{biosketch}

\begin{dedication}
\emph{For}
\end{dedication}

\begin{acknowledgements}
First and foremost, I would like to thank Dr. Ashish Raj and Dr. Srikantan Nagarajan, two advisors that provided opportunities and countless valuable advice from Weill Cornell to UCSF. Ashish always pushed the boundaries and was open to new things, Sri asked important questions and taught me how to critically approach the problems and tasks at hand. I am very appreciative of my thesis committee: Dr. Amy Kuceyeski, Dr. Keith Purpura, and Dr. Jonathan Victor, for sharing their expertise which was invaluable at every meeting. Thank you as well to Dr. Pablo F. Damasceno for accompanying me on pursuits of better data practices and mentorship advice.

I am extremely grateful for the Neuroscience Department at Weill Cornell Graduate School for providing a welcoming community and outstanding opportunities for teaching and peer led learning. Thanks to UCSF's Open Science Group, the Brainhack organizers, and Neurohackademy instructors, communities that kept me energized and shared valuable knowledge in scientific knowledge, publishing, ethics, and data practices. Thank you to the ReproNim team, INCF, and the ReproNim Fellowship trainers for broadening my horizons on the possibilities of scientific efforts. 

Thank you to my fellow scientists and friends at Weill Cornell for their expertise and support that got me through first-year courses. Thank you to my group of classmates and friends: Thomas, Rosa, Mitchell, and George who always showered me with warm welcomes whenever I visited from SF. Last, but not least, thank you to my Family. My parents have been incredibly supportive throughout my entire life, and I am lucky to have caring extended families in New York City, Beijing, and Chiba. To my parents, thank you for your sacrifices and for being there for me despite your busy work schedule.
%Finally, many scientists and friends have given selflessly their expertise and support, including: 
\end{acknowledgements}

\contentspage
\tablelistpage
\figurelistpage

\normalspacing \setcounter{page}{1} \pagenumbering{arabic}
\pagestyle{cornell} \addtolength{\parskip}{0.5\baselineskip}

% Abbreviations table
\section{Table of Abbreviations}
%\label{tab:abbreviations}
\begin{table}
\caption{Table of Abbreviations}
\centering
\begin{tabular}{l m{8cm} l}
	\toprule
	Akt & Protein Kinase B \\
	APC & Allophycocyanin\\
	\bottomrule
	\end{tabular}
	\label{tab:abbreviations}
\end{table}

%\subfile{abbreviations}

%\chapter{Introduction \& Background}
%\subfile{../chapter1/introduction}

\chapter{INTERFACIAL ACTIN PROTRUSIONS MECHANICALLY ENHANCE KILLING BY CYTOTOXIC T CELLS }
\label{chap:protrusions}
\subfile{../chapter2/chapter2}
%
%\chapter{Whole Brain Network Neural Mass Models: A Criticism}
%\label{chap:nmm}
%\subfile{../chapter3/wcnmm}
%
%\chapter{Emergence of Canonical Functional Networks from the Structural Connectome}
%\label{chap:lap}
%\subfile{../chapter4/chapter4_lap}
%
%\chapter{Spectral Graph Theory of Brain Oscillations}
%\label{chap:sgm}
%\subfile{../chapter5/chapter5_sgm}
%
%\appendix
%\chapter{Appendix: Supplementary Figures}
%\subfile{../appendix/supplements}

\bibliography{References}

\end{document}
