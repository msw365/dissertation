\documentclass[phd,tocprelim]{cornell}
%
% tocprelim option must be included to put the roman numeral pages in the
% table of contents
%
% The cornellheadings option will make headings completely consistent with
% guidelines.
%
% This template was originally provided by Blake Jacquot, and
% fixed up by Andrew Myers.
%
%Some possible packages to include
\usepackage[utf8]{inputenc}
\usepackage{graphicx,pstricks}
\usepackage{microtype}
\usepackage{graphics}
\usepackage{moreverb}
\usepackage{subfigure}
\usepackage{epsfig}
\usepackage{subfigure}
\usepackage{hangcaption}
\usepackage{txfonts}
\usepackage{palatino}
\usepackage{gensymb}
\usepackage{longtable}
% Packages added by Xihe:
\usepackage{subfiles}
\usepackage{bm}
\usepackage[font=small, labelfont=bf]{caption}
\usepackage{array} % for flexible tables
\usepackage{hyperref} % for links
\usepackage{amsbsy} % for bold
\usepackage{caption}
\usepackage{indentfirst} % indent first paragraph after \section tags
\usepackage{booktabs} % for tables
%Packages added by Mitchell
%

%if you're having problems with overfull boxes, you may need to increase
%the tolerance to 9999
\tolerance=9999

\bibliographystyle{unsrt}
%\bibliographystyle{IEEEbib}

\renewcommand{\caption}[1]{\singlespacing\hangcaption{#1}\normalspacing}
\renewcommand{\topfraction}{0.85}
\renewcommand{\textfraction}{0.1}
\renewcommand{\floatpagefraction}{0.75}

\title{MECHANOTRANSDUCTION OF THE CYTOTOXIC T LYMPHOCYTE  EFFECTOR RESPONSE}
\author {Mitchell S. Wang}
\conferraldate {May}{2022}
\degreefield {Ph.D.}
\copyrightholder{Mitchell S. Wang}
\copyrightyear{2022}

\begin{document}

\maketitle
\makecopyright

\begin{abstract}
\subfile{../abstract/abstract}
\end{abstract}

\begin{biosketch}
Mitchell was born in 1993 in Palo Alto, California and came of age in the suburbs of Philadelphia, Pennsylvania. After his high school education, he attended New York University, where he studied for his Bachelor of Arts degree as a double major in both biology and economics. During his time at NYU, he studied Alzheimer’s disease in the laboratory of Dr. Yueming Li at Memorial Sloan Kettering Cancer Center. Upon graduating in 2015 and with an interest in the biomedical sciences, he entered the Pharmacology program at Weill Cornell Graduate School of Medical Sciences and joined the laboratory of Dr. Morgan Huse at Memorial Sloan Kettering Cancer Center for his thesis work, where he has since studied the principles of mechanical force exertion in T cells and their contribution to cellular cytotoxicity.

Mitchell's best memories away from the lab during graduate school were spent in the warmth of Jacob Riis beach, in the colorful company of his friends he met in New York City, and behind the lens of his digital and film cameras.
\end{biosketch}

\begin{dedication}
\emph{for Mama, Baba, and Nick}
\end{dedication}

\begin{acknowledgements}
Graduate school simply cannot be done alone, and far be it from me to write this many words without mentioning the people who have helped make this dream come true. I must first thank my thesis advisor Dr. Morgan Huse for inviting an untrained, greenhorn scientist into his lab to work on visually and viscerally exciting science, where together we asked and answered groundbreaking questions about T cell biology and killing with creative and nearly artistic experiments. I must thank him for mentoring me throughout this doctoral process; training, developing, and eventually welcoming me to join the community of scientists. Thank you Morgan.

I would also like to thank my thesis committee members Drs. John Blenis, Philipp Niethammer, Michel Sadelain, and  David Scheinberg, for their scientific direction and suggestions - for a young scientist, the time of an experienced scientist is worth its weight in gold. 

Thank you to the collaboratoring labs and brilliant scientists who have made this scientific work possible and without whom these ideas would simply remain ideas. Thank you to the Dr. Lance Kam laboratory at Columbia University, in particular: Dr. Susie Jin, Dr. Parthiv Chaudhuri,  Dr. Joanne Lee, Dr. Chirag Sachar, and Xin Wang, who have helped open this world of T cell mechanobiology to me. The Dr. Khalid Salaita lab at Emory University as well, particularly Yuesong Hu, who made our manuscript way more exciting and our audiences way more awed. Thank you!

Of course, my dear labmates, past and present. This work is as much yours as it is mine - we did it! Thank you to Ben Whitlock, who showed many of us that graduate school was usually only as stressful of an experience as you let it be. To Fella, for allowing the younger me assist you on your protrusions paper. To Minggang, for always being around in the lab at 2 A.M. to show the bleary-eyed me where the bacteria spreaders are. To Diana, for reinforcing my belief that scientists can dress well and look stylish every day! To Alex, for your reassuring and free-flowing attitude to whatever today's strange topic of conversation is. To Maria, for your boundless ebullience and colorful patterned dresses. To Ben Winer, for your unflappable attitude towards science, humor, and your career - best of luck! To Lizzy, for your lively cheerfulness and always complimenting my appearance each day. To Ty, and his postdoc expertise (lol) - I hope you enjoy living in New Jersey so much. To Miguel, for all of your advice over the years: philosophical, scientific, musical, and personal. And of course to my lab sibling Elisa, who knows that my gratitude for her cannot be fully expressed in words. 

My friends in graduate school - thank you for sharing the wealth that is your lives with me. My cup runneth over! Living in New York City is magical, but only because the people make it so. Thank you to Mark, in helping me grow in every way, possible and impossible. Thank you to Thomas, Rosa, Alan, Bobby, and George for the countless hot pot nights and for providing safe haven during the most uncertain periods of the Pandemic. Special shoutout to Bobby for showing me the beauty of LaTeX and serving as a private IT service. Thank you to Louise, Kathleen, and Mariano - my memories of our times together (and to come) are lost somewhere in a blur of fantastic wine and cheese and the Reopening.

Last but not least, I must thank my family: my mom and my dad and my younger brother Nick and my extended family members around the globe in Germany and China, for always picking me up whenever I might fall. Happiness does not always come easy, but the support of my family always has, and I know always will!
\end{acknowledgements}

\contentspage
\tablelistpage
\figurelistpage

% Abbreviations table
\abbrlist
%\label{tab:abbreviations}
\begin{longtable}{| p{.30\textwidth} | p{.70\textwidth} |}
%\caption{Table of Abbreviations}
%\centering
%\begin{tabular}{l m{10cm} l}
%	\toprule
	ADAP, or Fyb & Adhesion- and degranulation-promoting adaptor protein \\	
	Akt & Protein Kinase B \\
	APC & Allophycocyanin \\
	APC & Antigen presenting cell \\
	Arp2/3 	& Actin related protein-2/-3 complex \\
	Bak & BCL2 antagonist/killer 1 \\
	Bax & BCL2 associated X \\
	BID & BH3 interacting-domain death agonist \\
	$Ca^{2+}$ & Calcium ion \\
	CAR & Chimeric antigen receptor \\
	CD3$\zeta$ & Cluster of differentiation 3 zeta-chain \\
	CD11a & Cluster of differentiation 11a \\
	CD28 & Cluster of differentiation 28 \\
	CD45 & Cluster of differentiation 45 \\
	CD69 & Cluster of differentiation 69 \\
	CD107a & Cluster of differentiation 107a, see Lamp-1 \\
	Cdc42 & Cell division control protein 42 homolog \\
	CLP & Common lymphoid progenitor cells \\
	CRAC &  Calcium release-activated channel \\
	CRISPR & Clustered regularly interspaced short palindromic repeats \\
	cSMAC & Central supramolecular activation cluster \\
	CTL & Cytotoxic T lymphocyte \\
	DAG & Diacylglycerol \\
	DMSO & Dimethylsulfoxide \\
	dSMAC & Distal supramolecular activation cluster \\
	Endoplasmic reticulum & ER \\	
	ERK  & Mitogen-activated protein kinase \\
	F-actin & Filamentous actin \\
	FITC & Fluorescein isothiocyanate \\
	G-actin & Globular actin \\
	GEF & Guanine nucleotide exchange factor \\
	GFP & Green fluorescent protein \\
	gRNA & Guide RNA for CRISPR \\
	GTPase & Guanosine triphosphatase \\
	Gzmb & Granzyme B \\
	H2Kb-OVA & OVA presented by the class I MHC protein \\
	HSC & Hemaetopoietic stem cells \\
	ICAM-1 & Intercellular adhesion molecule 1 \\
	IFN$\gamma$ & Interferon gamma \\
	I$\kappa$B & Inhibitor of kappa B \\
	IL-2 & Interleukin-2 \\
	ILP & Invadosome-like protrusion\\
	$IP_3$ & inositol 1,4,5-trisphosphate \\
	Inositol trisphosphate receptor & InsP3R \\
	iRFP670 & Infrared fluorescent protein 670 \\
	IS & Immunological synapse \\
	ITAM & Immunoreceptor tyrosine-based activation motif \\
	Lamp1 & Lysosomal-associated membrane protein 1 \\
	LAT & Linker for activation of T cells \\
	Lck & Lymphocyte-specific protein tyrosine kinase \\
	LDH	& Lactate dehydrogenase \\
	LFA-1 & Lymphocyte function-associated antigen 1 \\
	LG & Lytic granules \\
	MAPK & Mitogen-activated protein kinase \\
	MHC & Major Histocompatibility Complex \\
	MTOC & Microtubule organizing center \\
	MTP & Membrane tension probe \\
	Nck & Non-catalytic region of tyrosine kinase protein \\
	NF$\kappa$B & Nuclear factor kappa-light-chain-enhancer of activated B cells \\
	NK & Natural killer \\
	NPF & Nucleation promoting factor \\
	NT & Non-targeting \\
	OT-1 & TCR specific for the OVA peptide \\
	OVA & Ovalbumin 257-264 peptide (sequence: SIINFEKL) \\
	PAMP & Pathogen-associated molecular pattern \\
	PDMS & Polydimethylsiloxane \\
	PFA & Paraformaldehyde \\
	pHluorin & pH-sensitive GFP \\
	PI3K & Phosphoinositide 3-kinase \\
	$PIP_2$ & Phosphatidylinositol-4,5-bisphosphate \\
	$PIP_3$ & Phosphatidylinositol-3,4,5-trisphosphate \\
	PKC$\theta$ & Protein kinase C theta \\
	PLC$\gamma$ & Phospholipase gamma \\
	PMA & Phorbol 12-myristate 13-acetate \\
	PMAi & Phorbol 12-myristate 13-acetate and ionomycin \\
	pMHC & Peptide-MHC, see MHC \\
	pN & Piconewton \\
	pSMAC & Peripheral supramolecular activation cluster \\
	RAG & Recombination-activating gene \\
	Ras & Rat sarcoma virus protein \\
	RFP & Red fluorescent protein \\
	SA & Streptavidin \\
	STIM1 & Stromal interaction molecule 1 \\
	TCR & T cell receptor \\
	Vav1 & Vav Guanine Nucleotide Exchange Factor 1 \\
	WASP & Wiskott-Aldrich syndrome protein \\
	WAVE2 & WASP-verprolin homolog 2 \\
	WT & Wild-type \\
	YFP & Yellow fluorescent protein \\
%	\bottomrule
%	\end{tabular}
	\label{tab:abbreviations}
\end{longtable}
%\subfile{abbreviations}

\normalspacing \setcounter{page}{1} \pagenumbering{arabic}
\pagestyle{cornell} \addtolength{\parskip}{0.5\baselineskip}

\chapter{INTRODUCTION AND BACKGROUND}
\subfile{../chapter1/chapter1}

\chapter{INTERFACIAL ACTIN PROTRUSIONS MECHANICALLY ENHANCE KILLING BY CYTOTOXIC T CELLS }
\label{chap:protrusions}
\subfile{../chapter2/chapter2}

\chapter{MECHANICALLY ACTIVE INTEGRINS DIRECT CYTOTOXIC SECRETION AT THE IMMUNE SYNAPSE}
\label{chap:lfa1}
\subfile{../chapter3/chapter3}

\chapter{CONCLUSION}
\label{chap:conclusion}
\subfile{../chapter4/chapter4}
%
%\appendix
%\chapter{Appendix: Supplementary Figures}
%\subfile{../appendix/supplements}

\bibliography{References}

\end{document}
