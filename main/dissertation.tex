\documentclass[phd,tocprelim]{cornell}
%
% tocprelim option must be included to put the roman numeral pages in the
% table of contents
%
% The cornellheadings option will make headings completely consistent with
% guidelines.
%
% This template was originally provided by Blake Jacquot, and
% fixed up by Andrew Myers.
%
%Some possible packages to include
\usepackage[utf8]{inputenc}
\usepackage{graphicx,pstricks}
\usepackage{microtype}
\usepackage{graphics}
\usepackage{moreverb}
\usepackage{subfigure}
\usepackage{epsfig}
\usepackage{subfigure}
\usepackage{hangcaption}
\usepackage{txfonts}
\usepackage{palatino}
\usepackage{gensymb}
% Packages added by Xihe:
\usepackage{subfiles}
\usepackage{bm}
\usepackage[font=small, labelfont=bf]{caption}
\usepackage{array} % for flexible tables
\usepackage{hyperref} % for links
\usepackage{amsbsy} % for bold
\usepackage{caption}
\usepackage{indentfirst} % indent first paragraph after \section tags
\usepackage{booktabs} % for tables
%Packages added by Mitchell
\usepackage{gensymb} % for degree symbol
%Find a Mandarin one!

%if you're having problems with overfull boxes, you may need to increase
%the tolerance to 9999
\tolerance=9999

\bibliographystyle{unsrt}
%\bibliographystyle{IEEEbib}

\renewcommand{\caption}[1]{\singlespacing\hangcaption{#1}\normalspacing}
\renewcommand{\topfraction}{0.85}
\renewcommand{\textfraction}{0.1}
\renewcommand{\floatpagefraction}{0.75}

\title{MECHANOTRANSDUCTION OF THE CYTOTOXIC T LYMPHOCYTE  EFFECTOR RESPONSE}
\author {Mitchell S. Wang}
\conferraldate {May}{2022}
\degreefield {Ph.D.}
\copyrightholder{Mitchell S. Wang}
\copyrightyear{2022}

\begin{document}

\maketitle
\makecopyright

\begin{abstract}
\subfile{../abstract/abstract}
\end{abstract}

\begin{biosketch}
Mitchell was born in Palo Alto, California in 1993, and grew up in the suburbs of Philadelphia, Pennsylvania. After high school, he attended New York University, where he studied for his Bachelor of Arts degree as a double major in both biology and economics. During his time in college, he studied Alzheimer’s disease in the laboratory of Dr. Yueming Li at Memorial Sloan Kettering Cancer Center. Upon graduating in 2015 and with an interest in the biomedical sciences, he entered the Pharmacology program at Weill Cornell Graduate School of Medical Sciences and joined the laboratory of Dr. Morgan Huse at Memorial Sloan Kettering Cancer Center for his thesis work, where he has since studied the principles of mechanical force exertion in T cells and their contribution to cellular cytotoxicity.
\end{biosketch}

\begin{dedication}
\emph{for Mama, Baba, and Didi}
\end{dedication}

\begin{acknowledgements}
Morgan, collaborators, committee members, lab members, grad school friends, family
\end{acknowledgements}

\contentspage
\tablelistpage
\figurelistpage

\normalspacing \setcounter{page}{1} \pagenumbering{arabic}
\pagestyle{cornell} \addtolength{\parskip}{0.5\baselineskip}

% Abbreviations table
\section{Table of Abbreviations}
%\label{tab:abbreviations}
\begin{table}
\caption{Table of Abbreviations}
\centering
\begin{tabular}{l m{10cm} l}
	\toprule
	Akt & Protein Kinase B \\
	APC & Allophycocyanin \\
	APC & Antigen presenting cell \\
	Arp2/3 	& Actin related protein-2/-3 complex \\
	Bak & BCL2 antagonist/killer 1 \\
	Bax & BCL2 associated X \\
	BID & BH3 interacting-domain death agonist \\
	$Ca^{2+}$ & Calcium ion \\
	CAR & Chimeric antigen receptor \\
	CD3$\zeta$ & Cluster of differentiation 3 zeta-chain \\
	CD11a & Cluster of differentiation 11a \\
	CD28 & Cluster of differentiation 28 \\
	CD45 & Cluster of differentiation 45 \\
	CD69 & Cluster of differentiation 69 \\
	CD107a & Cluster of differentiation 107a, see Lamp-1 \\
	Cdc42 & Cell division control protein 42 homolog \\
	CLP & Common lymphoid progenitor cells \\
	CRISPR & Clustered regularly interspaced short palindromic repeats \\
	cSMAC & Central supramolecular activation cluster \\
	CTL & Cytotoxic T lymphocyte \\
	DAG & Diacylglycerol \\
	DMSO & Dimethylsulfoxide \\
	dSMAC & Distal supramolecular activation cluster \\
	ERK  & Mitogen-activated protein kinase \\
	F-actin & Filamentous actin \\
	FITC & Fluorescein isothiocyanate \\
	G-actin & Globular actin \\
	GEF & Guanine nucleotide exchange factor \\
	GFP & Green fluorescent protein \\
	gRNA & Guide RNA for CRISPR \\
	GTPase & Guanosine triphosphatase \\
	Gzmb & Granzyme B \\
	H2Kb-OVA & OVA presented by the class I MHC protein \\
	HSC & Hemaetopoietic stem cells \\
	ICAM-1 & Intercellular adhesion molecule 1 \\
	IFN$\gamma$ & Interferon gamma \\
	I$\kappa$B & Inhibitor of kappa B \\
	IL-2 & Interleukin-2 \\
	ILP & Invadosome-like protrusion\\
	$IP_3$ & inositol 1,4,5-trisphosphate \\
	iRFP670 & Infrared fluorescent protein 670 \\
	IS & Immunological synapse \\
	ITAM & Immunoreceptor tyrosine-based activation motif \\
	Lamp1 & Lysosomal-associated membrane protein 1 \\
	LAT & Linker for activation of T cells \\
	Lck & Lymphocyte-specific protein tyrosine kinase \\
	LDH	& Lactate dehydrogenase \\
	LFA-1 & Lymphocyte function-associated antigen 1 \\
	LG & Lytic granules \\
	MAPK & Mitogen-activated protein kinase \\
	MHC & Major Histocompatibility Complex \\
	MTOC & Microtubule organizing center \\
	MTP & Membrane tension probe \\
	Nck & Non-catalytic region of tyrosine kinase protein \\
	NF$\kappa$B & Nuclear factor kappa-light-chain-enhancer of activated B cells \\
	NK & Natural killer \\
	NPF & Nucleation promoting factor \\
	NT & Non-targeting \\
	OT-1 & TCR specific for the OVA peptide \\
	OVA & Ovalbumin 257-264 peptide (sequence: SIINFEKL) \\
	PAMP & Pathogen-associated molecular pattern \\
	PDMS & Polydimethylsiloxane \\
	PFA & Paraformaldehyde \\
	pHluorin & pH-sensitive GFP \\
	PI3K & Phosphoinositide 3-kinase \\
	PIP2 & Phosphatidylinositol-4,5-bisphosphate \\
	PIP3 & Phosphatidylinositol-3,4,5-trisphosphate \\
	PKC$\theta$ & Protein kinase C theta \\
	PLC$\gamma$ & Phospholipase gamma \\
	PMA & Phorbol 12-myristate 13-acetate \\
	PMAi & Phorbol 12-myristate 13-acetate and ionomycin \\
	pMHC & Peptide-MHC, see MHC \\
	pN & Piconewton \\
	pSMAC & Peripheral supramolecular activation cluster \\
	RAG & Recombination-activating gene \\
	Ras & Rat sarcoma virus protein \\
	RFP & Red fluorescent protein \\
	SA & Streptavidin \\
	TCR & T cell receptor \\
	Vav1 & Vav Guanine Nucleotide Exchange Factor 1 \\
	WASP & Wiskott-Aldrich syndrome protein
	WAVE2 & WASP-verprolin homolog 2 \\
	WT & Wild-type \\
	YFP & Yellow fluorescent protein \\
	\bottomrule
	\end{tabular}
	\label{tab:abbreviations}
\end{table}
%\subfile{abbreviations}

\chapter{INTRODUCTION AND BACKGROUND}
\subfile{../chapter1/chapter1}

\chapter{INTERFACIAL ACTIN PROTRUSIONS MECHANICALLY ENHANCE KILLING BY CYTOTOXIC T CELLS }
\label{chap:protrusions}
\subfile{../chapter2/chapter2}

\chapter{MECHANICALLY ACTIVE INTEGRINS DIRECT CYTOTOXIC SECRETION AT THE IMMUNE SYNAPSE}
\label{chap:lfa1}
\subfile{../chapter3/chapter3}

\chapter{CONCLUSION}
\label{chap:conclusion}
\subfile{../chapter4/chapter4}
%
%\appendix
%\chapter{Appendix: Supplementary Figures}
%\subfile{../appendix/supplements}

\bibliography{References}

\end{document}
