Immune cell-cell communication is canonically presented as an interlinked network of biochemical pathways through which immune cells process information. However, this characterization fails to consider the physical methods that immune cells employ in order to interact with their local environment and enhance their cellular function. The conversion of mechanical stimuli into biochemical responses is called mechanotransduction. Cytotoxic T lymphocytes (CTLs) are an ideal system for studying these techniques because they are both physically active and immunologically communicative. Additionally, they are biomedically important, playing crucial roles in homeostatic and therapeutically induced immune responses against both foreign pathogens and cancer, thus making them therapeutically urgent and interesting to study.

CTLs exert physical force against target cells, thereby straining target cell membrane and mechano-potentiating the pore-forming activity of perforin and overall effector response. However, it remained unclear what structures T cells formed in order to exert physical force, and what proteins or signaling molecules could be involved in this phenomenon. We found that T cells utilized actin-rich protrusions at the cell-cell interface against the target cell to distort the target cell membrane. The formation of protrusions was dependent on the actin nucleation promoting factor Wiskott-Aldrich Syndrome protein (WASP), and was necessary for physical distortions in the target cell membrane that amplified T cell cytotoxicity. These results mechanistically clarified how T cells exert physical force against their targets to amplify the pore-forming effects of perforin oligomerization in the target cell surface. However, it remained unknown how T cells synchronized biochemical signaling and mechanical signaling for optimized cytotoxicity.

Given that higher levels of T cell mechanical force exertion were correlated with higher levels of killing, we were curious to know if T cells used mechanical cues at the synapse to determine where to secrete their lytic granules into the intracellular space, thereby optimizing the release of the cytotoxic payload. Such a phenomenon could also reconcile how T cells avoid killing healthy bystander cells in a crowded cellular environment, by requiring physical engagement from antigen-presenting cells as a signal of cellular adjacency and safety. We found that T cells specifically degranulated near the engaged integrin receptor LFA-1 (lymphocyte function-associated antigen 1) that formed upon the detection of antigen. Fascinatingly, this relationship was dependent on the mechano-transductive behavior of LFA-1, as ablation of its force-generating capacity via talin inhibited LFA-1 force exertion and degranulation completely. The ability to convert mechanical stimuli into a degranulation signal was also broadly generalizable to other integrins. These results, in their totality, identify how T cells are able to secrete lytic granules near areas of force exertion, and clarify how T cells are able to clear diseased cells with minimal bystander killing.

Our studies describe the sophisticated mechanisms used by the T cell to mechanically optimize its cytotoxicity. We identified WASP-driven actin-rich protrusions as a major driver of force exertion against the target cell membrane to prime it for perforin insertion and pore formation. Additionally, we identified a new mechanical role for the integrin LFA-1 (and broadly, all other integrins) in establishing sites of cell-cell proximity and therefore permissive lytic granule release in order to safely and efficiently deposit the cytotoxic payload against the target cell. Altogether, these studies further clarify the mechanical pathways that CTLs utilize to engage with and interpret their immune environment, ultimately resulting in efficient clearance of infected and tumor cells.
