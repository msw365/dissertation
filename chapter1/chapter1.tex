\section{The immune system and T cells}
The majority of organisms have an immune system, defined as a biological network of processes that defend the host organism against foreign pathogens and disease, that is vital for maintaining homeostasis. For mammalian immune systems, this network is broadly broken into two major arms: the innate immune system and the adaptive immune system. The innate immune system responds initially to pathogens but is limited in the molecular scope of the threats it responds to, as it uses pathogen-associated molecular patterns (PAMPs) to trigger its activity, which are generally conserved. Therefore the innate immune system responds faster but generically to pathogens - which, while useful to the host for clearing the majority of infectious agents, means the innate immune system can be overwhelmed by pathogens that may have evolved to evade innate immune system processes. It is this circumstance that creates the need for an adaptive immune system.

The adaptive immune system responds much more slowly than the innate immune system, using this time to harvest and collect antigen and begin upramping of its antigen-specific molecular processes. The adaptive immune system is composed of two major types of cells (called lymphocytes): B cells and T cells. B cells (named for originating from the bone marrow) produce antibodies that specifically bind to antigens and identify their bound partners as foreign for destruction. T cells on the other hand (named for originating in the thymus) directly manage the cytotoxic activity of the adaptive immune system. T cells are composed of two main functional subtypes: CD8+ killer T cells (the primary focus of this thesis and heretofore referred to as any of the following: cytotoxic T lymphocytes (CTLs), cytotoxic T cells, CD8+ T cells) and CD4+ helper T cells. 

\section{T cell development}
T cells as a whole are defined by their expression of the T cell receptor (TCR), the receptor responsible for recognizing foreign antigenic peptide presented by the target cell via its Class I or Class II Major Histocompatibility Complex (MHC, or pMHC). The TCR is multi-subunit receptor that is composed of varying combinations of $\alpha$-, $\beta$-, $\gamma$-, $\delta$-, $\epsilon$-, and $\zeta$-chains, whose specific combination defines the unique sub-identity of the T cell, although the vast majority of T cells (95\%+) express the $\alpha$ and $\beta$ chains and are thus called $\alpha \beta$ T cells. There are a number of additional co-receptors that bind (either directly or indirectly) to the TCR. The most relevant to $\alpha \beta$ T cells are the CD8 or CD4 co-receptors, mentioned above. These co-receptors transduce differing signals within the specific T cell and additionally define the specificity of the $\alpha \beta$ T cell to the two different classes of MHC: CD8+ T cells bind to MHC Class I, while CD4+ T cells bind to MHC Class II. 

The ability of T cells to properly recognize and engage only with foreign antigen while ignoring self-peptides is tightly regulated and begins at a very early stage in immune system development. T cells derive from hematopoietic stem cells (HSCs) that originate in the bone marrow. HSCs then ultimately differentiate into common lymphoid progenitor cells (CLPs), which migrate to the thymus to ultimately differentiate into natural killer (NK), B, or T cells. 

The process of differentiating into T cells is primarily motivated by the need to create a functional TCR that does not react to self-antigen but does react to foreign antigen. As TCRs are made up of alpha and beta chains that are evolved to react to a wide range of possible antigens that an organism may encounter in its lifespan, T cell differentiation has been characterized in a stepwise manner that first begins with TCR-beta chain selection. T cells at this stage express an invariant pre-alpha chain called pre-T$\alpha$ that the varying beta chains (generated by VDJ recombination of the TCR-beta locus) attempt to form a stable binding partner with. Once an appropriate TCR-beta chain is identified as capable of stable binding to pre-T, the same process begins on the TCR-alpha chain against the now mature TCR-beta chain, generating a stable (but not necessarily functional) TCR.

Once a stable TCR heterodimer has been formed, the T cells must undergo a process of positive and negative selection. Positive selection involves presenting the T cells with self-antigen presented on MHC with the selection criteria of being able to bind with this complex. Those T cells that are able to bind to an MHC complex presenting self-antigen receive a survival signal, while those with TCRs that cannot MHCs do not receive this survival signal. Therefore, the body positively filters/selects for T cells with TCRs that can recognize MHCs, leaving those that cannot recognize MHCs to die off. 

After obtaining T cells that are capable of binding to MHC molecules, the next step is to select for T cells that do not auto-react to self-antigen, called negative selection. This ensures that T cells are tolerant of self-antigen and prevents auto-immunity conditions in the organism. In this stage of selection, T cells that bind to MHC presenting self-antigen and activate receive an apoptotic signal that leads to cell death. Negative selection therefore prunes the T cell population for T cells that can not only bind to MHC but do not react to self-antigen. The vast majority of thymocytes (98\%+) fail to pass positive and negative selection. Following positive and negative selection, the resulting T cell set can therefore bind to any antigen (presumably belonging to any foreign pathogen) so long that the antigen is distinct from the body’s self-antigens. 
After these stages of T cell development, these T cells (called naïve T cells) then exit the thymus and begin to circulate in the host, where they will spend their time surveying the host for disease and foreign antigens.

\section{TCR signaling and the CD8+ T cell immune synapse}
If a CD8+ T cell engages an infected cell presenting foreign peptide, the T cell will initiate a cascade of antigenic-specific intracellular signaling events. Upon ligation of the TCR to the pMHC complex, the CD3 proteins (CD3$\epsilon \gamma$ and CD3$\epsilon \delta$ heterodimers and a CD3$\zeta$ homodimer) bearing ITAM (immunoreceptor tyrosine-based activation motif) also bind to the TCR. The ITAMs on the CD3$\zeta$ get phosphorylated 

\section{T cell cytoskeleton}
aaa 

\section{T cell degranulation and target cell death}
Upon successful formation of the IS, T cells will coordinate their lytic granules 

\section{T cell mechanical force exertion}
Immune cell-immune cell interactions are typically described as a series of interrelated biochemical receptor-ligand signaling processes (and this introduction has done the same). While this characterization is not incorrect, it ignores the mechanical dimension of immune cell interactions, which have been identified as a significant hubs of cellular decision making. Immune cells have been demonstrated to have a number of biophysical sensing functions, including cancer cell surveillance and immune cell function and cytotoxicity.

The CTL immune synapse has been shown to be a physically dynamic structure, capable of exerting force against a resisting target or surface and 

\section{Thesis Aims}
The further successful development of immune cell therapies thus far has been and will continue to be dependent on a deeper understanding of basic T cell effector function. To this end, this thesis is chiefly focused on a molecular approach to studying the mechanotransduction of T cell effector function and the dynamic relationships between its biophysical and biochemical dimensions. The results of this study are broken across two chapters, whose contents are summarized below.

While it was known that the CTLs use physical force to amplify the pore-forming effects of perforin and resulting positive correlation between force exertion and  target cell death, it was unclear what cellular structures T cells form in order to actualize this force against the target cell membrane and manipulate this relationship. It was also unclear what proteins or molecules were involved in this membrane perturbation/distortion process. Chapter 2 of this thesis will address the results of the experiments testing these outstanding questions through the use of pharmacological and genetic perturbations against the actin nucleation promoting factors (NPFs) Wiskott–Aldrich Syndrome protein (WASP) and WAVE2 (WASP-verprolin homolog 2) that directly affected T cell force exertion, and studying their resulting cytotoxic capabilities and dynamics via co-culture assays and imaging.

Another outstanding question arose from the observation that T cells degranulate specifically near areas of synaptic force exertion. While this dovetails with the observation that force exertion amplifies the pore-forming effects of perforin (as discussed in the Introduction and Chapter 2), it was unknown what types of forces or interactions coordinated these two phenomena of degranulation and synaptic force exertion so that they could be unified in space and time. It was also unknown what force-sensitive molecules could be mediating this intercellular communication. Chapter 3 addresses the study conducted to address these questions, which also used pharmacological and genetic perturbations against integrins (primarily lymphocyte function-associated antigen 1, or LFA-1) on T cells in order to modify the force output of these T cells and change their degranulation and cytotoxic behavior as observed through co-culture assays and unique and creative imaging experiments. Altogether, this thesis aims to clarify the specifics of the T cell killing event (encompassing T cell force exertion and degranulation) in order to better inform future T cell studies and therapy design.
