\section{Conclusion} 

For my doctoral work, I performed research investigating the mechanoregulation of cytotoxic T lymphocyte killing. Altogether, we found that CTLs form synaptic actin protrusions against target cells \cite{Tamzalit2018}. This act sensitizes the target cell to perforin lysis and granzyme entry, heightening the overall cytotoxic effector response. This potentiation is molecularly controlled by the actin nucleation promoting factors WASP and WAVE2, which are concentrically organized (from the center to the periphery, respectively). WASP promotes central actin protrusions associated with lytic granule fusion and content release, while WAVE2 is linked with peripheral actin protrusions and robust conjugate formation. Phenotypically, the increased application of force against targets is associated with an overall better effector response. This biomechanical force-effector relationship again appears in our model of degranulation, in which T cells use mechanical signals via integrins to home their lytic granules towards synaptic 'hotspots' in a novel manner distinct from their canonical signaling capacity through focal adhesions. These synaptic sites of granule fusion are first verified to belong to a foreign or infected cell via spatially adjacent TCR and LFA-1 engagement (see Figure \ref{fig:fig2twobeads}) and then are accordingly sensitized to perforin (see Figure \ref{fig:introddp}) for safe and secure killing. Although our data supports this model and is in line with published work in the field, there are key outstanding questions to be answered. A discussion of some of these questions and how the results of my thesis work might inform future studies follows.

\subsection{Integrin signaling in focal adhesions}

We established a necessary mechanical requirement for integrin-mediated degranulation, but much work remains to be done in characterizing how other molecules in focal adhesion machinery contribute to degranulation. Vinculin is a cytoskeletal molecule that consists of a head region that binds to talin or $\alpha$-actinin, and a tail region that binds to f-actin, lipids, and paxillin \cite{Goldmann2002}. Vinculin is regulated by phosphoinositide expression and reinforces the connection between talin and the f-actin cytoskeleton, stabilizing the mechanical action on integrins and talin \cite{Ezzell1997, Goldmann2002}. As the disruption of vinculin activity is known to affect the capacity of T cells to form proper focal adhesions and adhere \cite{Goldmann2002}, it would be a good place to begin investigation of focal adhesion machinery to degranulation using genetic means of perturbation and measuring force exertion and cytotoxic capacity using similar methods and techniques as described in Chapter \ref{chap:chapter3} (e.g. CRISPR, DNA MTP force experiments, TCR signaling, etc.). Of course, any of the other components of focal adhesions are also fair questions for study.

In Chapter \ref{chap:chapter3}, we hypothesized that the outside-in mechanical activity of LFA-1 drove lytic granule fusion at the synapse. However, one outstanding question is the respective contributions of inside-out vs. outside-in signaling of integrins to degranulation homing. As we used methods that would abolish both inside-out and outside-in signaling, experiments that only perturb the outside-in signaling are necessary. One possible method could involve the rescue of inside-out signaling after talin knock out via CRISPR by overexpressing the talin-head domain only which has been carefully characterized in the literature to rescue inside-out signaling \cite{Ciobanasu2018, Elliott2010, Ellis2014}. The talin head domain will bind to the cytosolic tail of the $\beta$-integrin domain and trigger inside-out signaling (see \ref{fig:lfa} for further detail) but be unable to bind the actin cytoskeleton, which is mediated by the tail domain. We predict that T cells that have been depleted for talin but express the talin-head domain will have active conformation LFA-1 on the surface and be able to adhere and form synapses, but be unable to home and fuse lytic granules at the synapse. Other alternative approaches could involve the use of the mangnese ion ($Mn^{2+}$) , which is known to stabilize the MIDAS domain of LFA-1 \cite{Meijne1994, Sen2018, Dixit2011}, to stabilize the LFA-1/ICAM-1 function and yield more mechanical force, possibly enhancing degranulation.

\subsection{Optimizing degranulation to escape T cell suicide}
An outstanding question in the T cell field is how T cells avoid the cytotoxic effects of their own perforin and achieve unidirectional killing \cite{Lopez2013}. Some studies search for particular shielding molecules expressed only on the T cell side of the synapse \cite{Balaji2002} or a broader T cell shielding method \cite{Rudd-schmidt} from perforin. In lieu of a particular molecule, we would like to suggest that T cells use a particular step-wise logic in order to confirm that the queried target that they are engaged with is a \textit{bona fide} target cell via physical force engagement (quite literally, hugging or embracing their target before applying \textit{le baiser de la mort}, or the kiss of death). Once a safe contact against a \textit{bona fide} target cell has been established, the T cell can then secrete perforin into a hyperlocal, perforin-sensitive synaptic cleft environment for target cell destruction. Physical force is critical for mediating this hypothetical function. 

The observation that T cells tend to granulate at pillar tops (see Figure \ref{fig:fig2degranulation}) aligns with existing knowledge in the field that lytic granules require actin hypodense/cleared regions for membrane fusion \cite{Ritter2015}.  However, the transient burst of f-actin associated with the moment of degranulation (Figure \ref{fig:fig2degranulation}) is ripe to consider as a mode of perforin shielding.  It could be associated with a burst of local mechanopotentiating synaptic force,  generating actin-flow mediated traction force to potentiate integrin granule homing (Figure \ref{fig:fig4model}). It could also be involved in actively polymerizing f-actin polymers, creating actin-rich micro-protrusions/structure at a resolution scale below detection via confocal microscopy.  Many common probes for actin detection would be insufficient for answering this question. Lifeact-GFP only detects mature f-actin as opposed to actively polymerizing actin, while phalloidin can only be used in fixed tissues and cells, rendering any dynamic behavior unobservable. Instead, one could use FRET-based actin polymerization probes such as g-actin-CFP in combination with g-actin-YFP to study dynamic actin polymerization, using the appearance of FRET as an indicator of actively polymerizing f-actin \cite{LansingTaylor1981}. These probes could be used \textit{in vitro} in many of the nanofabricated systems described in this thesis to interrogate if T cells were actively creating microdomains of g-actin polymerization into f-actin at sites of degranulation \cite{Okamoto2004}. Other potential strategies could involve photobleaching f-actin and monitoring fluorescent recovery of f-actin (a method known as fluorescence recovery after photobleaching, or FRAP) \cite{Simon1988} relative to where cells degranulate, or using a fluorescent WASP, WAVE2, or Arp2/3 probe to presumably monitor the relative location of actin nucleation promoting factors to lytic granule secretion \cite{Obino2016, Tamzalit2018}. 

One should also consider the local lipid archictecture of the T cell degranulation site at the base of actin-rich protrusions. There is existing work that conceptually backs a lipid-based line of reasoning for perforin resistance. One group identified that higher lipid order at the plasma membrane conferred increased resistance, and that exposed phosphatidylserine was capable of inactivating perforin \cite{Ruddschmidt}. Given the the concave orientation of the protrusion 'pit' relative to the corresponding target cell surface, we hypothesize that the perforin molecule preferentially binds to convex lipid membrane curvatures, due to its asymmetric, "key-like" sterics \cite{Law2010}. We could directly test this hypothesis by adapting gramicidin-based techniques from studies on fluorescence-based small unilamellar vesicles (SUVs) \cite{Ingolfsson2010,  Polozov2001}. Lipids will be constituted in buffer containing the fluorophore 8-Aminonaphthalene-1,3,6-Trisulfonic Acid (ANTS) and one of its quenchers p-Xylene-Bis-Pyridinium Bromide (DPX) to reform into large multilamellar vesicles (LMVs). Preference for convexity will be tested by treating the fluorescent SUVs with perforin, triggering oligomerization with calcium in the reaction buffer, and then measuring the amount of liberated fluorescence as a marker for pore formation. We expect the group in which perforin engages a convex lipid membrane to liberate more fluorophore than the group in which perforin engages a concave lipid membrane. Preference for concavity will be tested by constituting perforin into SUVs and triggering oligomerization and fluorophore leakage using A23187, a small molecule calcium ionophore (this ionophore can similarly be added to the convexity sample for full control).  Should it bear out that perforin has a preference for convex lipid membranes, it might potentially explain how T cells avoid being killed by their own secreted perforin, which remains an outstanding question in the field.

\subsection{SNARE complex crosstalk to integrins}
SNARE proteins are a large family of proteins that are responsible for mediating vesicle fusion and are directly responsible for the fusion of lytic granules \cite{Chang2017}.  SNARE proteins are localized to the vesicle membrane or target membranes and are respectively called v- and t-SNARES \cite{Yoon2018}. Synaptobrevin2 is the v-SNARE and syntaxin11 is the t-SNARE that mediates lytic granule fusion in cytotoxic lymphocytes \cite{Halimani2014, Chitirala2019, Matti2013}.  Some SNARE proteins are localized on both vesicle and target membranes, and so newer naming methods classify some v-SNAREs as R-SNAREs and vSNAREs as Q-SNAREs. Q- and R-SNAREs are named for respective amino acid (glutamine or arginine) they contribute to the zero ionic layer, the main site of intereaction in the SNARE complex. The $H_{abc}$ domain of the Q-SNARE domain folds back and allows the interaction between the v-SNARE and the t-SNARE, forming a \textit{trans-}SNARE complex \cite{Yoon2018}.  The SNARE activating function is regulated by Munc18 \cite{Spessott2017, Yoon2018, Baker2015}. Finally, the R-SNARE binds to the trimeric complex, forming a mature SNARE complex that works as a molecular zipper to bring the vesicle and plasma membranes closer together for fusion \cite{Agostino2017}.

It is possible that the finely tuned LFA-1 mediated mechanical homing of degranulation is communicated to the t-/v-SNARE complex, as SNARE complexes themselves are mechanosensitive \cite{Risselada2020}. One group observed that neuronal SNARE complexes placed under tension (around 10 pN) can maintain a partially-assembled form in which the v- and t-SNARES are in a 'frayed' state \cite{Zhou2017}. The observed mechanoregulation of SNARE complexes suggests that their mechanical activity can lead to increased biological activity (higher formation of the fusion stalk and vesicle fusion), but has yet to be directly tested in immune/T cells. There are reports of functional associations between the SNARE complex and integrin-mediated recycling at the membrane \cite{Riggs2012, Zhuang2020, Nagy2009}, but whether or not any signaling molecules liase between the two signalosomes remains unclear and would be a fecund area for future study. 

%To investigate possible integrin mechano-crosstalk to the SNARE complex, we could investigate the Munc-family group of proteins, known regulators of t-/v-SNARE fusion \cite{Ma2011, Yoon2018}. Possible candidates to study are the Munc13-4 and 18-2 adaptor proteins. While these proteins are known to be in direct control of the assembly of SNARE complex components in CTLs \cite{Diao2010}, not much is known about their specific regulation. Munc13-4 lacks the C1 domain that is common to most Munc-family proteins \cite{Rhee2002} but does contain several putative phosphorylation sites \cite{Neeft2005}, making it possible that Munc13-4 is regulated by force-dependent phosphorylation. Whether or not it is could be tested by knocking out talin in T cells (using the gRNA we characterized in Chapter \ref{chap:chapter3}) and profiling the phosphorylation of the Munc13-4 protein against T cells transduced with a non-targeting control gRNA in a phospho-proteomic screen (this would also interrogate whether or not outside-in signaling from integrins affects Munc13-4 function and SNARE fusion). 

%The regulation of Munc18-2 is also not well characterized.  One known general method of regulation of Munc-family protein activity is the localization of diacylglycerol at the plasma membrane. One could track the localization of Munc18-2 at the synapse (using fluorescent probes) in response to a photoactivated TCR \cite{Huse2007}, which is known to induce diacylglycerol localization \cite{Quann2009}. If there is crosstalk, photoactivation of the TCR should induce Munc18-2 and integrin localization. One could even test the influence of mechanical activity on Munc18-2 localization (and by proxy, SNARE localization) by generating photoactivatable agonists on peptides of different TCR signal strengths (i.e. different primary sequences), which are known to induce different mechanical strengths \cite{VanderMerwe2001, Ma2010, Ma2012, Lanzavecchia1999}. These experiments would be illuminating to see if mechano-regulation of integrins directly affects SNARE complex formation and function for lytic granule fusion.

\subsection{TCR signaling strength and integrin-mediated degranulation}

The mechanical activity of integrins is regulated by TCR signaling strength, as  force experiments in our lab and others on force-measuring surfaces without stimulatory peptide have indicated \cite{Ma2012, Feng2018, Chen2013, Harrison2019}. Different peptides presented in MHC are capable of stimulating different TCR signaling strengths as well \cite{Moran2011}. These observations lead us to ask how TCR signaling strengths affect integrin clustering and if there is a minimal number of integrin clusters needed for degranulation. Experimental techniques designed to ask the same question of minimal number of TCRs needed for T cell activation could be adapted to answer this question, such as restricting the clustering of integrins to test minimal requirements for degranulation \cite{Hashimoto-Tane2016, Labrecque2001, Manz2011}. 

Another outstanding question is how physiologically relevant the OT-1/OVA-SIINFEKL peptide system is, given its high TCR signaling strength \cite{Moran2011}. For this question, one could repeat the experiments outlined in Chapter \ref{chap:chapter2} and Chapter \ref{chap:chapter3} using a model peptide system more similar to what T cells encounter \textit{in vivo}. However, given that mechanical defects in human cells also produce lowered immune responses (such as in Wiskott-Aldrich syndrome or HIV \cite{Houmadi2018}), we expect the results to be similar.

\subsection{Fas-ligand mediated cytotoxicity}

Although the T cell primarily kills through the secretion of perforin and granzyme \cite{Yasukawa2000}, T cells have other ways in killing target cells as well. A well-known alternative pathway is the Fas ligand (FasL) pathway.  Binding of FasL against the Fas receptor (FasR, or CD95) leads to the formation of a trimeric "death-inducing signaling complex"(DISC), or FasL-FasR complex \cite{Yamada2017}. Upon the aggregation of DISCs and corresponding FasL-FasR downstream signaling, caspases 2, 3, 8, and 9 are released leading to DNA degradation, membrane blebbing, and apoptosis \cite{Parlato2000, Yamada2017}. Fascinatingly, Fas ligand has been observed to localize to cytotoxic granules containing perforin and granzyme \cite{Kojima2002}, with some groups even reporting a distinct granule subsets that contain Fas but no perforin or granzymes \cite{Chang2017, Schmidt2011, Schmidt2011_2}. 

From the perspective of this study,  a natural question to consider is whether or not the secretion of subpopulations of granules is segregated mechanically and for what purpose they might be. Probably the first pair of questions to investigate is whether or not Fas ligand or Fas receptor signaling is mechanically regulated, which could be studied using the techniques used in this study. One study has observed co-immunoprecipitation of FasR with ezrin, a known linker molecule to the actin cytoskeleton, suggesting that FasR could be mechanically regulated \cite{Parlato2000}. To directly test this hypothesis, one could imagine an experiment involving placing Fas-receptor positive target cells expressing fluorescent probes for the Fas-receptor mediated apoptosis pathway (e.g. caspases) onto traction force microscopy-compatible micropillars (thinner). These pillars could be coated with adhesive proteins and Fas ligand, which could trigger FasL-mediated apoptosis. Colocalization studies of caspases against areas of target cell force exertion could be informative.

\section{Future directions}

This thesis research attempted to comprehensively cover the mechanobiology of cytotoxicity (Chapter \ref{chap:chapter2}) and lytic granule secretion (Chapter \ref{chap:chapter3}), which are important, but ultimately only one region of the T cell field of study. What follows is an informed discussion of the future of mechanobiology as it pertains to \textit{other} T cell fields of study, and how the field of mechanobiology might change in the years to come.

\subsection{Population-scale CTL force dynamics}

Going forward, it would be interesting to investigate dynamics of synaptic force exertion in a single CTL against a target cell. In particular, one wonders if the CTL deploys a standard range of force exertion and protrusion-type or cellular structure at first pass (in a "probing phase"), and then must dynamically change its range of force exertion or arsenal of structures for force exertion according to local biostructural architecture (i.e. what is the lifetime force action of a standard CTL encounter?). Certainly these events are influenced by the biochemical activity of the signaling molecules present at the synapse. Answering such questions would require direct interrogation of a live CTL-target cell synapse. Previous technologies for measuring membrane tension include atomic force microscopy or an optical trap - but both remain insufficiently precise to monitor real-time \textit{in vitro} or \textit{in vivo} synapse dynamics. Certain fluorescence-based technologies have been recently been developed in order to measure synapse dynamics that ignore restrictive planar geometries. One group has developed spherical deformable hydrogel particles with tuneable stiffnesses that can be used to monitor force dynamics in a wide variety of ligand and stiffness contexts on the target surface. \cite{Vorselen2020}. Combined with fluorescent probes, the use of these microparticles could elucidate "phases" of the T cell physical response as it (putatively) probes, adheres, degranulates, kills, and releases, the target cell.

Thinking outside the context of single-T cell force exertions, it would be fascinating to investigate the host CTL-ome at large, if certain "battalions" or "legions" of CTLs exist, organized according to their physical strengths and effector capabilities. Certainly a reasonable first-pass method of organization is expression levels of certain effector genes and molecules (e.g. \textit{Prf}, \textit{Grzb}, \textit{Ifng}), but these measures obviously omit the  mechanoregulation of these molecules and thus miss their effects in totality. One would require experimental techniques capable of not only measuring single-cell T cell force exertion \textit{en masse}, but also the possible swarm behavior of responding T cells in which an initial "less capable" group of T cells is replaced by stronger effector cells (or weaker, in a "clean up" phase). Certain molecular perturbation platforms that are designed to measure CRISPR-based perturbations at the protein level \cite{Wroblewska2018} come closer to characterizing all of the effects that a molecular perturbation truly encompasses. Other platforms for monitoring infiltrating immune cells into locally perturbed regions of the tumor microenvironment also exist \cite{MaximeDhainaut2021}. One could imagine choosing a tumor system for this platform that has been spatially perturbed heterogeneously for known cytoskeletal/cell stiffness molecules and investigate how T cells of well-characterized strengths segregate (and presumably kill).

\subsection{The mechanobiology of exhausted T cells}

To date, there has been no direct study of mechanobiology of exhausted T cells. Nevertheless, given what we know about the immunological behavior of exhausted T cells and how mechanobiological principles influence these behaviors in other T cell contexts (such as na{\"i}ve, proliferating, or activated T cells), we can comfortably speculate on how we might think mechanical principles might influence exhausted T cell (dys)function.

The label 'exhausted T cells' can be quite wide, but general agreement can be found in a definition that includes a molecular signature of increased expression of inhibitory receptors, such as: PD-1 (programmed cell death protein 1), lymphocyte-activation gene 3 (LAG), T-cell immunoglobulin and mucin-domain containing-3 (TIM3), and T cell immunoreceptor with Ig and ITIM domains (TIGIT), as well as characteristic transcription factors such as B lymphocyte induced maturation protein 1 (Blimp-1), basic leucine zipper transcription factor, ATF-like (BATF), nuclear factor of activated T cells (NFAT), T-box expressed in T cells (T-bet), and eomesodermin (Eomes) \cite{Wherry2011, Quigley2010, Keir2008, Doering2012, Wherry2007, Wang2012,  Lazarevic2013, Rangachari2012, Shin2009, Paley2012, Larsson2013}. Although direct evidence is needed, it seems reasonable to suggest that exhausted T cells might display diminished physical capacity, further dampening their effector capabilities. It is likely, as TCR mechanical signaling is a known requirement for T cell activation \cite{Hu2016}. In particular, exhausted T cells are strongly associated with a $PD-1^{hi}$ phenotype \cite{Keir2008}. Persistent antigen stimulation in the context of chronic infection triggers epigenetic alternations to the \textit{Pdcd1} locus, resulting in this higher surface expression of the PD-1 inhibitory receptor. The PD-1/PD-L1 signaling axis is known to attenuate TCR activation \cite{Mizuno2019}, proliferation \cite{Schietinger2014}, and T cell migration \cite{Zinselmeyer2013}. 

Exhausted T cells also present a diminished metabolic state. Accordingly, it is likely that their effector capacity is severely weakened, considering that cytoskeletal remodeling is a highly energy intensive activity \cite{Bernstein2002}. Reports on the energy dynamics of actin-polymerization inhibited cells indicate that ATP consumption is up to 50\% slower \cite{Bernstein2002, Ahmed2015}. Depletion of glucose/ATP from the medium also immobilizes the motor protein myosin \cite{Xu2014}, which is known to be involved in potentiating lytic granule fusion \cite{Basu2016}.

\subsection{On nutrient mechano-sensing}

Nutrient sensing is a necessary activity for cell survival and is a meticulously regulated process. mTOR (the mammalian target of rapamycin) is a master regulator of nutrient sensing and receives significant research interest \cite{Park2020}. It is a kinase that upon activation promotes anabolism and macromolecule synthesis. mTOR can be found on lysosomes \cite{Rabanal-Ruiz2018} and localizes to focal adhesions \cite{Rabanal-Ruiz2018}. Focal adhesions are major nutrient-sensing sites/hubs \cite{Hamidi2021} because of their proximity to  endo-/exosomal membrane fusion-related traffic. In this sense, it is significantly advantageous for mTOR to position itself at the membrane in order to sense and metabolize amino acids \cite{Shimobayashi2016} and other nutrients.

Since FAs are mechanically active sites of mTOR localization, and mTOR is in close proximity to talin in focal adhesions (\textless 10nm) \cite{Rabanal-Ruiz2021}, an exciting possibility to consider is that mTOR is mechanically driven to home towards focal adhesions. The questions raised in the subsection \textit{Integrin signaling in focal adhesions} reappear - what mechanosensitive molecules are homing mTOR to focal adhesions? What molecules dynamically regulate the level of mTOR activity with corresponding mechanical force? These are all exciting potential lines of investigation. The purpose of such an arrangement within T cells might be to ensure contact with a foreign and dying cell, so that its nutrients might be scavenged or used for mTOR signaling, but this is only speculative.

\subsection{Mechanically enhancing the success of engineered T cells}

Much work remains to be done to achieve the full potential of CARs as a successful and accessible therapy. Despite significant public and private sector effort in designing and developing novel cancer immunotherapies, these efforts are almost ubiquitously underappreciate the biomechanical aspects of T cell killing, which play a large role in their cytotoxicity. In addition to their effector function, T cells participate in mechanical communication throughout their entire lifecycle, surveying organs and tissues for foreign cells and experiencing both harsh and forgiving physical environments such as the shear force of lymphatic transport or the broad range tissue stiffnesses that exist within the body. These biomechanical properties are significantly altered in disease states. T cells take significant cues from their mechanical environment in their activity, altering their development, differentiation, migration, activation, and effector functions, all of which can tip the balance of successful vs. unsuccessful immune system clearance of pathogens or cancers. Biomechanical communication has been extensively studied, but has yet to be significantly incorporated in immunotherapy design for both knowledge and practical reasons. Genetic modification of the cancer cell is exceedingly difficult, and the expensive process of \textit{ex vivo} modification of the T cells affects accessibility to the therapy for cancer patients.

Nonetheless, growing knowledge of T cell mechanobiology has enabled the  creation of mechanically-based engineering strategies to modulate T cell immunity, a nascent field termed “mechanical immunoengineering”. Mechanical immunoengineering takes advantage of biomechanical cues such as stiffness and external forces in order to modulate T cell proliferation and effector functions for therapeutic applications. Of course, this is in complement to well-established modes of modulating biochemical cues such as antigen recognition (chimeric antigen receptors) and co-stimulatory signals (checkpoint blockade) that curate the anti-tumor T cell immune response and enhance therapeutic efficacy. For example, stiffer antigen-presenting matrices have been shown to enhance T cell proliferation independently of the intensity of biochemical stimulatory signals \cite{Lei2020} and can be adapted and used \textit{ex vivo} in order to improve patient outcomes. Fabrication of planar surfaces presenting T cell ligands has demonstrated that focal clustering of TCR can be inhibited by presenting antigens in a hollow ring configuration, which in turn reduces IFN$\gamma$ and proliferation of T cells \cite{Schraml2015}. In contrast, TCR clustering signaling can be promoted by introducing regions of ligand for multi-valent receptor binding on the activating surface, which boosts the effector function of T cells \cite{OConnor2012, Schraml2015}.  These studies suggest that spatial organization of the TCR clusters control the T cell response,  which should strongly inform the design of these immunoengineering strategies \cite{Aramesh2019}.

There has been some study of the CAR immune synapse, although it remains in the preliminary stages and the relationship between the unique organization of the CAR IS and CAR efficiency have yet to be fully explored. Most strikingly, the CAR IS does not present an annular bull's eye structure, which is a characteristic feature of TCR IS.  The organization of the actin ring in CAR IS is poor, and actin may not be not completely diminished at the center of CAR IS \cite{Xiong2018}, which may disrupt smooth granule secretion. LFA-1 is disorganized at the CAR IS, and ligand-bound CAR microclusters are randomly distributed \cite{Davenport2019}. Finally, the CAR IS induces significant CAR-proximal signaling very quickly (\textless 2 min), suggesting that CARs may not need to fully form stable IS structures for effective cytotoxicity \cite{Li2020, Watanabe2018}. 

A recent report details the development of a CAR T cell that upon activation, secretes an enzyme that cleaves adminstered pro-drugs for localized drug administration \cite{Gardner2021}. This platform could be locally tuned by placing the enzymatic gene of interest under the control of a mechano-sensitive gene transcription factor. This strategy would render the therapeutic activity of the drug sensitive to the local stiffness of the tumor microenvironment, accompanying the likely mechanical potentiation of CAR T cell activation and effector function. Of course, one could combine this approach with drugs designed to stiffen the tumor microenvironment (perhaps with certain crosslinking agents) for increased efficacy in mechanopotentiating CAR T cells. 

\subsection{Nuclear mechanotransduction}

The T cell nucleus is mechanosensitive as well, which is not surprising given its substantial size in T cells \cite{McGregor2016}, taking up almost 90\% of the cell volume \cite{Strokotov2009}. T cell forces are mechanotransduced through the lamina, a network of lamin and lamin-binding proteins that are anchored to the actin cytoskeleton \cite{Rossy2018, Enyedi2017, Dahl2008}. When T cells migrate, squeezing and stretching can deforms not only the plasma membrane, but also the nucleus, which triggers mechano-sensitive receptors located on the nuclear membrane \cite{Dahl2008, Guilluy2014}. 

The mechanical tension applied to these receptors modify the heterochromatin state of cellular DNA, and transcription activity of the T cell \cite{Le2016}. This mechanical tension can also trigger the proinflammatory eicosanoid pathway \cite{Enyedi2017}, which is typically activated in wound healing contexts. Eicosanoids are proinflammatory signaling molecules that regulate a number of T lymphocyte functions \cite{Lone2013}. There is significant therapeutic focus in targeting these pathways, and future research may be benefit from incorporating mechanical dimensions of T cell activation into their therapeutic strategies. 

\subsection{Limitations of \textit{in vitro}-based study of T cells}

Many of the experiments conducted in order to complete this doctoral thesis were performed using \textit{in vitro} systems. These systems were designed to interrogate force activity by holding the force threshold constant (e.g. micropillars, DNA membrane tension probes), while observing the cell's force activity in time. While these arrangements are useful for ease of investigation, holding constant a variable as broad as the force threshold creates major scientific blind spots even in real-time studies (much less bulk studies via co-culture assays). This is true particularly in studies focused on the effector cell, in which the arrangement ignores half of the cellular conversation: the target cell \cite{Tello-lafoz2021}. The target cell is capable of dynamically responding to the effector cell's efforts to kill (presumably in a cooperative or antagonistic manner), which is entirely neglected in these experimental systems.  More studies of cooperative target cell mechanical contributions to killing are needed.

\section{\textit{Fin.}}

A number of outstanding questions surrounding T cell mechanobiology remain open for investigation. These questions remain unanswered today for either knowledge or technically-based reasons, but they will surely be answered in due time.
