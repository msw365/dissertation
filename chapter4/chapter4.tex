For my doctoral work, I performed research investigating the mechanoregulation of cytotoxic T lymphocyte killing. We found that CTLs form synaptic actin protrusions against target cells. This act sensitizes the target cell to perforin lysis and granzyme entry, heightening the overall cytotoxic effector response. This potentiation is molecularly controlled by the actin nucleation promoting factors WASP and WAVE2, which are concentrically organized (from the center to the periphery, respectively). WASP promotes central actin protrusions associated with lytic granule fusion and content release, while WAVE2 is linked with peripheral actin protrusions and robust conjugate formation. Phenotypically, the increased application of force against targets is associated with an overall better effector response. This biomechanical force-effector relationship again appears in our model of degranulation, in which T cells use mechanical signals via integrins to home their lytic granules towards synaptic 'hotspots'. These synaptic sites of granule fusion are verified to be foreign and accordingly sensitized to perforin. 

A number of outstanding questions surrounding T cell mechanobiology remain open for investigation. A discussion of some of these questions and how the results of my thesis work informs these questions follows. These questions remain unanswered today for either knowledge or technically-based reasons, but they will surely be answered in due time.

\section{Mechanical licensing of cellular activities}

Many experimental systems that interrogate 

- Importance of mechanobiology in developmental biology

- Other speculative integrins

- Other speculative functions that mechanical activity can contribute to

\section{Interfacing mechanical and biochemical signaling planes}

- Different antigen sensitivities / valency or sensitivity of integrins? Minimal number needed for degranulation?

- Speculate on particular moleucles or signaling axes

- How do integrin focal adheisons isgnal to fusion machinery? How do integrins speak to the centrosome?

- Are there local lipid domains at regions of integrin mechanosensitivity? https://rupress.org/jcb/article/221/1/e202112057/212925/Thank-ORP9-for-FFAT-With-endosomal-ORP10-it-s

\section{Mechanical signaling in T cell subsets and engineered T cells}

- Exhausted T cells

- Speculation on the role of mechanobiology in engineered T cells

- Other immune cell types


Ideas:
- Compare to neuronal firing

- "Many immune receptors are mechanosensitive"

Formats:
- Based on this paper, it seems plausible that...

- It remains unknown 

Appendix ideas:

- Relevance of our in vitro system? Stiffnesses? The limitations of nterrogating a moleucle's biomechanology independently of its biochemistry?