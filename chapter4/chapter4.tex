For my doctoral work, I performed research investigating the mechanoregulation of cytotoxic T lymphocyte killing. Together, we found that CTLs form synaptic actin protrusions against target cells \cite{Tamzalit2018}. This act sensitizes the target cell to perforin lysis and granzyme entry, heightening the overall cytotoxic effector response. This potentiation is molecularly controlled by the actin nucleation promoting factors WASP and WAVE2, which are concentrically organized (from the center to the periphery, respectively). WASP promotes central actin protrusions associated with lytic granule fusion and content release, while WAVE2 is linked with peripheral actin protrusions and robust conjugate formation. Phenotypically, the increased application of force against targets is associated with an overall better effector response. This biomechanical force-effector relationship again appears in our model of degranulation, in which T cells use mechanical signals via integrins to home their lytic granules towards synaptic 'hotspots'. These synaptic sites of granule fusion are verified to be foreign and accordingly sensitized to perforin. 

A number of outstanding questions surrounding T cell mechanobiology remain open for investigation. A discussion of some of these questions and how the results of my thesis work informs these questions follows. These questions remain unanswered today for either knowledge or technically-based reasons, but they will surely be answered in due time.

\section{On the interfacing of mechanical and biochemical signaling planes}

- Different antigen sensitivities / valency or sensitivity of integrins? Minimal number needed for degranulation?

- Speculate on particular moleucles or signaling axes

- How do integrin focal adheisons isgnal to fusion machinery? How do integrins speak to the centrosome?

- Are there local lipid domains at regions of integrin mechanosensitivity? https://rupress.org/jcb/article/221/1/e202112057/212925/Thank-ORP9-for-FFAT-With-endosomal-ORP10-it-s

\section{On CTL force dynamics}

It would be interesting to investigate dynamics of synaptic force exertion in a single CTL against a target cell. In particular, one wonders if the CTL at first pass deploys a standard range of force exertion and protrusion-type or cellular structure (in a "probing phase"), and then must dynamically change its range of force exertion or arsenal of structures for force exertion according to local biostructural architecture (i.e. what is the lifetime force action of a standard CTL encounter?). Certainly these events are correspondingly influenced by the signaling molecules present at the synapse, each of which live their own biophysical lives by mere virtue of their presence. Answering such questions would require direct interrogation of a live CTL-target cell synapse. Previous technologies for measuring membrane tension include atomic force microscopy or an optical trap - but both remain insufficiently precise to monitor real-time \textit{in vitro} or \textit{in vivo} synapse dynamics. Certain fluorescence-based technologies have been recently been developed in order to measure synapse dynamics that ignore restrictive planar geometries \cite{Vorselen2020}. Combined with fluorescent probes, the use of these microparticles could elucidate "phases" of the T cell physical response as it (putatively) probes, adheres, degranulates, kills, and releases, the target cell.

Thinking outside the context of single-T cell force exertions, it would be fascinating to investigate the host CTL-ome at large, if certain "battalions" or "legions" of CTLs exist, organized according to their physical strengths and effector capabilities. Certainly a reasonable first-pass method of organization is expression levels of certain effector genes and molecules (e.g. \textit{Prf}, \textit{Grzb}, \textit{Ifng}), but these measures obviously omit the  mechanoregulation of these molecules and thus miss their effects in totality. One would require  experimental techniques capable of not only measuring single-cell T cell force exertion \textit{en masse}, but also the possible swarm behavior of responding T cells in which an initial "less capable" group of T cells is replaced by stronger effector cells (or weaker, in a "clean up" phase). Certain molecular perturbation platforms that are designed to measure CRISPR-based perturbations at the protein level \cite{Wroblewska2018} come closer to characterizing all of the effects that a molecular perturbation truly encompasses. Other platforms for monitoring infiltrating immune cells into locally perturbed regions of the tumor microenvironment also exist \cite{MaximeDhainaut2021}. One could imagine choosing a tumor system that has been spatially perturbed heterogeneously for known cytoskeletal/cell stiffness molecules and investigate how T cells of well-characterized strengths segregate (and presumably kill). Characterizing T cell swarm or group behavior would be a massive undertaking that would take many years to perform.

Many of the experiments conducted in order to complete this doctoral thesis were performed using \textit{in vitro} systems. These systems were designed to interrogate force activity by holding the force threshold constant (e.g. micropillars, DNA membrane tension probes), while observing the cell's force activity in time. While these arrangements are useful for ease of investigation, holding constant a variable as broad as the force threshold creates major scientific blind spots even in real-time studies (much less bulk studies via co-culture assays). This is true particularly in studies focused on the effector cell, in which the arrangement ignores half of the cellular conversation: the target cell \cite{Tello-lafoz2021}. The target cell is capable of dynamically responding to the effector cell's efforts to kill (presumably in a cooperative or antagonistic manner), which is entirely neglected in these experimental systems. 

\section{On the mechanical licensing of cellular activities}


- Importance of mechanobiology in developmental biology

- Other speculative integrins

- Other speculative functions that mechanical activity can contribute to

\section{On mechanical signaling of T cell subsets and engineered T cells}

Given how energy intensive 

- Exhausted T cells

- Speculation on the role of mechanobiology in engineered T cells

- Other immune cell types

\section{On the limitations of \textit{in vitro}-based investigations}

Of course, many of these experiments were also performed using the OT-I transgene system. 


Ideas:
- Compare to neuronal firing

- "Many immune receptors are mechanosensitive"

Formats:
- Based on this paper, it seems plausible that...

- It remains unknown 

