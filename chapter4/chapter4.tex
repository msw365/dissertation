For my doctoral work, I performed research investigating the mechanoregulation of cytotoxic T lymphocyte killing. Together, we found that CTLs form synaptic actin protrusions against target cells \cite{Tamzalit2018}. This act sensitizes the target cell to perforin lysis and granzyme entry, heightening the overall cytotoxic effector response. This potentiation is molecularly controlled by the actin nucleation promoting factors WASP and WAVE2, which are concentrically organized (from the center to the periphery, respectively). WASP promotes central actin protrusions associated with lytic granule fusion and content release, while WAVE2 is linked with peripheral actin protrusions and robust conjugate formation. Phenotypically, the increased application of force against targets is associated with an overall better effector response. This biomechanical force-effector relationship again appears in our model of degranulation, in which T cells use mechanical signals via integrins to home their lytic granules towards synaptic 'hotspots' in a novel manner that is distinct from their canonical signaling capacity through focal adhesions. These synaptic sites of granule fusion are immunologically verified to be foreign and accordingly sensitized to perforin. 

A number of outstanding questions surrounding T cell mechanobiology remain open for investigation. A discussion of some of these questions and how the results of my thesis work informs these questions follows. These questions remain unanswered today for either knowledge or technically-based reasons, but they will surely be answered in due time.

\section{On the biomechanical intersectionality}

- Different antigen sensitivities / valency or sensitivity of integrins? Minimal number needed for degranulation?

- Speculate on particular moleucles or signaling axes

- How do integrin focal adheisons isgnal to fusion machinery? How do integrins speak to the centrosome?

- Are there local lipid domains at regions of integrin mechanosensitivity? https://rupress.org/jcb/article/221/1/e202112057/212925/Thank-ORP9-for-FFAT-With-endosomal-ORP10-it-s

\section{On CTL force dynamics}

It would be interesting to investigate dynamics of synaptic force exertion in a single CTL against a target cell. In particular, one wonders if the CTL at first pass deploys a standard range of force exertion and protrusion-type or cellular structure (in a "probing phase"), and then must dynamically change its range of force exertion or arsenal of structures for force exertion according to local biostructural architecture (i.e. what is the lifetime force action of a standard CTL encounter?). Certainly these events are correspondingly influenced by the signaling molecules present at the synapse, each of which live their own biophysical lives by mere virtue of their presence. Answering such questions would require direct interrogation of a live CTL-target cell synapse. Previous technologies for measuring membrane tension include atomic force microscopy or an optical trap - but both remain insufficiently precise to monitor real-time \textit{in vitro} or \textit{in vivo} synapse dynamics. Certain fluorescence-based technologies have been recently been developed in order to measure synapse dynamics that ignore restrictive planar geometries \cite{Vorselen2020}. Combined with fluorescent probes, the use of these microparticles could elucidate "phases" of the T cell physical response as it (putatively) probes, adheres, degranulates, kills, and releases, the target cell.

Thinking outside the context of single-T cell force exertions, it would be fascinating to investigate the host CTL-ome at large, if certain "battalions" or "legions" of CTLs exist, organized according to their physical strengths and effector capabilities. Certainly a reasonable first-pass method of organization is expression levels of certain effector genes and molecules (e.g. \textit{Prf}, \textit{Grzb}, \textit{Ifng}), but these measures obviously omit the  mechanoregulation of these molecules and thus miss their effects in totality. One would require  experimental techniques capable of not only measuring single-cell T cell force exertion \textit{en masse}, but also the possible swarm behavior of responding T cells in which an initial "less capable" group of T cells is replaced by stronger effector cells (or weaker, in a "clean up" phase). Certain molecular perturbation platforms that are designed to measure CRISPR-based perturbations at the protein level \cite{Wroblewska2018} come closer to characterizing all of the effects that a molecular perturbation truly encompasses. Other platforms for monitoring infiltrating immune cells into locally perturbed regions of the tumor microenvironment also exist \cite{MaximeDhainaut2021}. One could imagine choosing a tumor system that has been spatially perturbed heterogeneously for known cytoskeletal/cell stiffness molecules and investigate how T cells of well-characterized strengths segregate (and presumably kill). Characterizing T cell swarm or group behavior would be a massive undertaking that would take many years to perform.

Many of the experiments conducted in order to complete this doctoral thesis were performed using \textit{in vitro} systems. These systems were designed to interrogate force activity by holding the force threshold constant (e.g. micropillars, DNA membrane tension probes), while observing the cell's force activity in time. While these arrangements are useful for ease of investigation, holding constant a variable as broad as the force threshold creates major scientific blind spots even in real-time studies (much less bulk studies via co-culture assays). This is true particularly in studies focused on the effector cell, in which the arrangement ignores half of the cellular conversation: the target cell \cite{Tello-lafoz2021}. The target cell is capable of dynamically responding to the effector cell's efforts to kill (presumably in a cooperative or antagonistic manner), which is entirely neglected in these experimental systems. 

\section{On mechanical licensure}

- Importance of mechanobiology in developmental biology

- Other speculative integrins

- Other speculative functions that mechanical activity can contribute to

\section{On the mechanobiology of exhausted T cells}

To date, there has been no direct study of mechanobiology of exhausted T cells. Nevertheless, given what we know about the immunological behavior of exhausted T cells and how mechanobiological principles influence these behaviors in other T cell contexts (such as na{\"i}ve, proliferating, or activated T cells), we can comfortably speculate on how we might think mechanical principles might influence exhausted T cell (dys)function.

The label 'exhausted T cells' can be quite wide, but general agreement can be found in a definition that includes a molecular signature of increased inhibitory receptor expression, such as B lymphocyte induced maturation protein 1 (Blimp-1), basic leucine zipper transcription factor, ATF-like (BATF), nuclear factor of activated T cells (NFAT), T-box expressed in T cells (T-bet), eomesodermin (Eomes), and PD-1 (programmed cell death protein 1) \cite{Wherry2011, Quigley2010, Doering2012, Wherry2007, Wang2012, Rangachari2012, Shin2009, Paley2012, Larsson2013}. In particular, exhausted T cells are strongly associated with a $PD-1^{hi}$ phenotype. Persistent antigen stimulation in the context of chronic infection triggers epigenetic alternations to the \textit{Pdcd1} locus, resulting in this higher surface expression of the PD-1 inhibitory receptor. The PD-1/PD-L1 signaling axis is known to attenuate TCR activation \cite{Mizuno2019}, proliferation \cite{Schietinger2014}, and T cell migration \cite{Zinselmeyer2013}. Although direct evidence is needed, it seems reasonable to suggest that exhausted T cells might display diminished physical capacity, further dampening their effector capabilities.

TCR mechanical signaling is a known requirement for T cell activation \cite{Hu2016}.

Exhausted T cells also present a diminished metabolic state. Their effector capacity is likely severely weakened as well, considering that cytoskeletal remodeling is a highly energy intensive activity. Reports on the energy dynamics of actin-polymerization inhibited cells indicate that ATP consumption is up to 50\% slower \cite{Bernstein2002, Ahmed2015}. Depletion of glucose/ATP from the medium also immobilizes the motor protein myosin \cite{Xu2014}, which is known to be involved in potentiating lytic granule fusion \cite{Basu2016}.

\section{On the mechanobiology of engineered T cells}

Despite the intense global research effort in developing novel cancer immunotherapies, a major dimension of the interactions between cancer and the immune system, its biomechanical aspect, has been largely underappreciated. Throughout their lifecycle, T cells constantly survey a multitude of organs and tissues and experience diverse biomechanical environments, such as shear force in the blood flow and a broad range of tissue stiffness. Furthermore, biomechanical properties of tissues or cells may be altered in disease and inflammation. Biomechanical cues, including both passive mechanical cues and active mechanical forces, have been shown to govern T cell development, activation, migration, differentiation, and effector functions. In other words, T cells can sense, respond to, and adapt to both passive mechanical cues and active mechanical forces. Biomechanical cues have been intensively studied at a fundamental level but are yet to be extensively incorporated in the design of immunotherapies. Nonetheless, the growing knowledge ofT cell mechanobiology has formed the basis for the development of novel engineering strategies to mechanically modulate T cell immunity, a nascent field that we termed “mechanical immunoengineering”. Mechanical immunoengineering exploits biomechanical cues (e.g., stiffness and external forces) to modulate T cell differentiation, proliferation, effector functions, etc., for diagnostic or therapeutic applications. It provides an additional dimension, complementary to traditional modulation of biochemical cues (e.g., antigen density and co-stimulatory signals), to tailor T cell immune responses and enhance therapeutic outcomes. For example, stiff antigen-presenting matrices have been shown to enhance T cell proliferation independently of the intensity of biochemical stimulatory signals \cite{Lei2020}

Engineering of ligands on planar surfaces has shown thatTCR arrangement can be manipulated by external cues.It was demonstrated that focal clustering of TCR can beinhibited by presenting antigens in a hollow ring config-uration, which in turn reduces IFN-gsecretion and pro-liferation of T-cells[42]. In contrast, TCR clustering andcell signaling can be promoted by introducing multivalentareas of receptors on the activating surface[42,43]. Allthesestudies suggest that TCR clusters cancontrol T-cellresponse by their micro and nanoscale organization \cite{Aramesh2019}

The formation of CAR IS has characteristics unlike the structure of TCR IS. The CAR IS does not present a systematic bull's eye structure, which is a characteristic feature of TCR IS. Organization of the actin ring in CAR IS is poor and actin may not be not completely diminished at the center of CAR IS (22). LFA-1 is disorganized and CAR-tumor antigen complexes form microclusters that are randomly distributed at the CAR IS (23) (Figure 1B). While TCR IS requires 5–10 min to form the bull's eye structure, the CAR IS might not need to form these stable structures because the disorganized multifocal pattern of CAR IS is sufficient to rapidly induce significant proximal signaling, which occurs within a short period of time (<2 min). Another important part of IS biology is the delivery of cytotoxic granules, including perforin and granzymes, to the IS mediated by microtubule organizing center (MTOC) (24). The rapid but short duration of proximal signaling of CAR IS also induces rapid MTOC migration to the IS and accelerates the delivery of granules (23). Although the mechanisms of CAR IS have gradually been revealed, it is still unclear whether the differences in CAR IS structure correlate with the efficacy of CAR T cells \cite{Li2020, Watanabe2018}.

\section{On \textit{in vitro}-based investigations}

Of course, many of these experiments were also performed using the OT-I transgene system. 


Ideas:
- Compare to neuronal firing

- "Many immune receptors are mechanosensitive"

Formats:
- Based on this paper, it seems plausible that...

- It remains unknown 

